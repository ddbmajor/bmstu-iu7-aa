\chapter{Конструкторская часть}
В данном разделе будут рассмотрены схемы алгоритмов сортировок (вставками, пузырьком и гномья), а также найдена их трудоемкость

\section{Требования к ПО}
Ряд требований к программе: на вход подается массив целых чисел в диапазоне от -10000 до 10000, возвращается отсортированный по месту массив, который был задан в предыдущем пункте. \newline

\section{Разработка алгоритмов}
На рисунках 2.1, 2.2 и 2.3 представлены схемы алгоритмов сортировки - вставками, пузырьком и гномья

\img{170mm}{insert_scheme.png}{Схема алгоритма сортировки вставками}
\img{200mm}{bubble_scheme.png}{Схема алгоритма сортировки пузырьком}
\img{220mm}{gnome_scheme.png}{Схема алгоритма гномьей сортировки}

\clearpage

\section[Модель вычислений для проведения оценки трудоемкости]{Модель вычислений для проведения\\ оценки трудоемкости}
Введем модель вычислений \cite{model}, которая потребуется для определния трудоемкости каждого отдельно взятого алгоритма сортировки:

\begin{enumerate}
	\item операции из списка (\ref{for:operations}) имеют трудоемкость равную 1:
	\begin{equation}
		\label{for:operations}
		\begin{gathered}
			+, -, /, *, \%, =, +=, -=, *=, /=, \%=, \\
			==, !=, <, >, <=, >=, [], ++, {-}-;
		\end{gathered}
	\end{equation}
	\item трудоемкость оператора выбора \code{if условие then A else B} рассчитывается, как:
	\begin{equation}
		\label{for:if}
		f_{if} = f_{\text{условия}} +
		\begin{cases}
			f_A, & \text{если условие выполняется,}\\
			f_B, & \text{иначе;}
		\end{cases}
	\end{equation}
	\item трудоемкость цикла рассчитывается, как:
	\begin{equation}
		\label{for:cycle}
		f_{for} = f_{\text{инициализации}} + f_{\text{сравнения}} + N(f_{\text{тела}} + f_{\text{инкремент}} + f_{\text{сравнения}})\text{;}
	\end{equation}
	\item трудоемкость вызова функции равна 0.
\end{enumerate}


\section{Трудоёмкость алгоритмов}

Определим трудоемкость выбранных алгоритмов сортировки по коду.

\subsection{Алгоритм сортировки пузырьком}

Трудоёмкость в лучшем случае (при уже отсортированном массиве):
\begin{equation}
	\label{for:bubble_best}
    f_{best} = 2 + 1 + N \cdot (1 + 1 + 1 + \frac{N - 1}{2} \cdot (1 + 1 + 4)) = 3N^2 + 3 = O(N^2).
\end{equation}

Трудоёмкость в худшем случае (при массиве, отсортированном в обратном порядке):
\begin{equation}
	\label{for:bubble_worst}
	\begin{gathered}
    	f_{worst} = 2 + 1 + N \cdot (1 + 1 + 1 + \frac{N - 1}{2} \cdot (1 + 1 + 11)) = \\
		= 6.5N^2 - 3.5N + 3 = O(N^2).
	\end{gathered}
\end{equation}

\subsection{Алгоритм гномьей сортировки}

Трудоёмкость в лучшем случае (при уже отсортированном массиве):
\begin{equation}
	\label{for:gnome_best}
    f_{best} = 1 + N(4 + 1) = 5N + 1 = O(N)
\end{equation}

Трудоёмкость в худшем случае (при массиве, отсортированном в обратном порядке):
\begin{equation}
	\label{for:gnome_worst}
    f_{worst} = 1 + N(4 + (N - 1) * (7 + 2)) = 9N^2 - 5N + 1 = O(N^2)
\end{equation}


\subsection{Алгоритм сортировки вставками}
Трудоемкость данного алгоритма может быть рассчитана с использованием той же модели подсчета трудоемкости.

Трудоемкость алгоритма сортировки вставками: в лучшем случае - $O(N)$, в худшем случае - $O(N^2)$ \cite{insert-sort}. \newline

\section*{Вывод}

Были разработаны схемы всех трех алгоритмов сортировки. Также для каждого из них были рассчитаны и оценены лучшие и худшие случаи.

\chapter{Аналитическая часть}
В этом разделе будут рассмотрены алгоритмы сортировок - вставками, пузырьком, гномья.

\section{Сортировка вставками}
\textbf{Сортировка вставками} \cite{insert-sort} - размещение элемента входной последовательности на подходящее место выходной последовательности.

Набор данных условно разделяется на входную последовательность и выходную. В начале отсортированная часть пуста. Каждый $i$-ый элемент, начиная с $i = 2$, входной последовательности размещается в уже отсортированную часть до тех пор, пока изначальные данные не будут исчерпаны.


\section{Сортировка пузырьком}
\textbf{Сортировка пузырьком} \cite{bubble-sort} - Алгоритм сортировки “пузырьком” состоит в повторении проходов по массиву с помощью вложенных циклов. При каждом проходе по массиву сравниваются между собой пары “соседних” элементов. Если элементы какой-то из сравниваемых пар расположены в неправильном порядке – происходит обмен (перезапись) значений ячеек массива. Проходы по массиву повторяются $N-1$ раз.


\section{Гномья сортировка}
\textbf{Гномья сортировка} \cite{gnomme-sort} - алгоритм сортировки, который использует только один цикл, что является редкостью.

В этой сортировке массив просматривается селва-направо, при этом сравниваются и, если нужно, меняются соседние элементы. Если происходит обмен элементов, то происходит возвращение на один шаг назад. Если обмена не было - агоритм продолжает просмотр массива в поисках неупорядоченных пар.

\section*{Вывод}
В данной работе необходимо реализовать алгоритмы сортировки, описанные в данном разделе, а также провести их теоритическую оценку и проверить ее экспериментально.
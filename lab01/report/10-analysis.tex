\titleformat{\section}[hang]
{\bfseries\normalsize}{\thesection}{1em}{}
\titlespacing\section{\parindent}{\parskip}{\parskip}
\section{Аналитическая часть}

\subsection{Цель и задачи}

Цель: изучение метода динамического программирования на материале расстояний Левенштейна и Дамерау-Левенштейна.

Для поставленной цели требуется решить следующие задачи:

\begin{enumerate}[leftmargin=1.6\parindent]
	\item разработать алгортимы поиска расстояний Левенштейна и Дамерау-Левенштейна и изучить термины.
	\item Реализовать алгортимы поиска расстояний Левенштейна и Дамерау-Левенштейна.
	\item Выполнить оценку реализованных алгоритмов по памяти.
	\item Выполнить замеры процессорного времени работы реализованных алгортимов.
	\item Выполнить сравнительный анализ нерекурсивных алгортимов поиска расстояний Левенштейна и Дамерау-Левенштейна.
	\item Выполнить сравнительный анализ алгортимов поиска расстояния \newline Дамерау-Левенштейна (итерационный, рекурсивный, рекусивный с кэшем).
\end{enumerate}
\pagebreak
\subsection{Обзор существующих алгоритмов}

\subsubsection{Расстояние Левенштейна}
Пусть строки s1 = s1[1\dots l1], s2 = s2[1\dots l2], где\newline
l1, l2 - длины строк\newline
А, s1[1\dots i] -- подстрока s1 длиной i\newline
s2[1\dots j] -- подстрока s2 длиной j

Тогда расстояние Левенштейна:\newline
\begin{equation}
	D=
	\begin{cases}
		0, i=0, j=0\\
		j, i=0, j>0\\
		i, i>0, j=0\\
		min\Bigl(D(\text{s1[1\dots i], s2[1\dots j-1]}) + 1,\\
		\;\;\;\;\;\;\;\;D(\text{s1[1\dots i-1], s2[1\dots j]}) + 1,\\
		\;\;\;\;\;\;\;\;D(\text{s1[1\dots i-1], s2[1\dots j-1]}) + 
		\begin{cases}
			0, s1[i] = s2[i]\\
			1, \text{иначе}
		\end{cases}
		\Bigr), \\\text{иначе}.
	\end{cases}
\end{equation}
\subsubsection{Расстояние Дамерау-Левенштейна}
Расстояние Дамерау-Левенштейна:\newline
\begin{equation}
	D=
	\begin{cases}
		0, i=0, j=0\\
		j, i=0, j>0\\
		i, i>0, j=0\\
		min\Bigl(D(\text{s1[1\dots i], s2[1\dots j-1]}) + 1,\\
		\;\;\;\;\;\;\;\;D(\text{s1[1\dots i-1], s2[1\dots j]}) + 1,\\
		\;\;\;\;\;\;\;\;D(\text{s1[1\dots i-2], s2[1\dots j-2]}) + 1\\
		\;\;\;\;\;\;\;\;D(\text{s1[1\dots i-1], s2[1\dots j-1]}) + 
		\begin{cases}
			0, s1[i] = s2[i]\\
			1, \text{иначе}
		\end{cases}\Bigr),\\ i>1, j>1, s1[i-1]=s2[j-2], s2[j-1]=s1[i-2]\\
		min\Bigl(D(\text{s1[1\dots i], s2[1\dots j-1]}) + 1,\\
		\;\;\;\;\;\;\;\;D(\text{s1[1\dots i-1], s2[1\dots j]}) + 1,\\
		\;\;\;\;\;\;\;\;D(\text{s1[1\dots i-1], s2[1\dots j-1]}) + 
		\begin{cases}
			0, s1[i] = s2[i]\\
			1, \text{иначе}
		\end{cases}
		\Bigr),\\\text{иначе}.
	\end{cases}
\end{equation}
\subsubsection{Общие подходы}
Существуют несколько подходов к реализации алгоритма поиска расстояний Левенштейна и Дамерау-Левенштейна.\par
Иттерационный подход: начиная с левого верхнего угла матрицы, описывающей расстояния между подстроками исхоных строк, 
считаются последующие значения до правого-нижнего угла, где и будет записано искомое растояние.\par
\pagebreak
Рекурсивный подход: ищется сразу искомое расстояние с использованием рекусивных вызовов функции от необходимых подстрок.\par
У этого метода существует недостаток: придется несколько раз пересчитывать уже вычисленные растояния.\par
Рекурсивный подход с кешем: подход аналогичен с рекусивным но также имеется матрица, в которой хранятся уже вычисленные расстояния.
При каждом вызове проверяется не вычисленно ли уже искомое расстояние. Это существенно ускоряет процесс.

\pagebreak
\titleformat{\section}[hang]
{\bfseries\normalsize\filcenter}{\thesection}{1em}{}
\titlespacing\section{\parindent}{\parskip}{\parskip}
\section*{ВВЕДЕНИЕ}
\addcontentsline{toc}{section}{ВВЕДЕНИЕ}

Для решения задачи о сравнении двух строк Левенштейн предложил ввести редакционное расстояние, которое будет учитывать количество 
редакционных операций дляя перехода от одной строки к другой.\par 
Расстояние Левенштейна -- минимальное количество операций, необходимых для преобразования одной строки в другую.\par
{\bf Редакционные операции:}
\begin{itemize}
    \item вставка 1 символа (insert);
    \item удаление 1 символа (delete);
    \item замена 1 символа (replace).
\end{itemize}

Существует несколько алгортимов, реализующих расчет расстояния Левенштейна.\par
Дамерау предложил добавить в число редакционных операций перестановку двух соседних символов, так как было замечено, что одна
из самых частых ошибок, допускаемых при наборе текста, является перестановка двух соседних букв. 

\pagebreak
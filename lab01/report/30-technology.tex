\section{Технологическая часть}

\subsection{Выбор языка программирования, средств разработки}
Для реализации алгоритмов был выбран язык C(c99). 
В данной лабораторной работе необходимо замерить процессорное время,
такую возможность дает стандартная библиотека \textbf{time.h}\cite{c99}.
Компилятор - \textbf{GCC}\cite{gcc}.
\subsection{Реализация алгоритмов}
В листингах 1-4 представлены реализации алгоритмов нахождения
расстояния Левенштейна и Дамерау-Левенштейна.
\pagebreak
\begin{code}
	\captionsetup{justification=centering}
	\captionof{listing}{Иттерационный алгоритм поиска расстояния Левенштейна}
	\label{code:1}
	\inputminted
	[
	frame=single,
	framerule=0.5pt,
	framesep=20pt,
	fontsize=\small,
	tabsize=4,
	linenos,
	numbersep=5pt,
	xleftmargin=10pt,
	]
	{text}
	{code/l.c}
\end{code}
\pagebreak
\begin{code}
	\captionsetup{justification=centering}
	\captionof{listing}{Иттерационный алгоритм поиска расстояния Дамерау-Левенштейна}
	\label{code:2}
	\inputminted
	[
	frame=single,
	framerule=0.5pt,
	framesep=20pt,
	fontsize=\small,
	tabsize=4,
	linenos,
	numbersep=5pt,
	xleftmargin=10pt,
	]
	{text}
	{code/d.c}
\end{code}
\begin{code}
	\captionsetup{justification=centering}
	\captionof{listing}{Рекурсивный алгоритм поиска расстояния Дамерау-Левенштейна}
	\label{code:3}
	\inputminted
	[
	frame=single,
	framerule=0.5pt,
	framesep=20pt,
	fontsize=\small,
	tabsize=4,
	linenos,
	numbersep=5pt,
	xleftmargin=10pt,
	]
	{text}
	{code/dr.c}
\end{code}
\pagebreak
\begin{code}
	\captionsetup{justification=centering}
	\captionof{listing}{Рекурсивный с кэшем алгоритм поиска расстояния Дамерау-Левенштейна}
	\label{code:4}
	\inputminted
	[
	frame=single,
	framerule=0.5pt,
	framesep=20pt,
	fontsize=\small,
	tabsize=4,
	linenos,
	numbersep=5pt,
	xleftmargin=10pt,
	]
	{text}
	{code/drc.c}
\end{code}

\subsection{Тестирование}
В таблице 1 приведены тесты для функций, реализующих алгоритмы поиска расстояний Левенштейна и Дамерау-Левенштейна. Тесты для всех
алгоритмов пройдены успешно.\par
\begin{table}[h]
	\caption{Проведенные тесты}
    \begin{tabular}{|c|c|c|c|c|}
    \hline
    №  & Строка 1 & Строка 2 & Результат Левенштейн & Результат Дамерау-Л. \\ \hline
    1  & \textquotedbl\textquotedbl       & \textquotedbl\textquotedbl       & 0                    & 0                    \\ \hline
    2  & \textquotedbl\textquotedbl       & mama     & 4                    & 4                    \\ \hline
    3  & babushka & \textquotedbl\textquotedbl       & 8                    & 8                    \\ \hline
    4  & moloko   & oloko    & 1                    & 1                    \\ \hline
    5  & baton    & at       & 3                    & 3                    \\ \hline
    6  & remont   & pat      & 5                    & 5                    \\ \hline
    7  & rov      & kit      & 3                    & 3                    \\ \hline
    8  & tovar    & to       & 3                    & 3                    \\ \hline
    9  & stakan   & satkan   & 2                    & 1                    \\ \hline
    10 & aabb     & abab     & 2                    & 1                    \\ \hline
    \end{tabular}
    \end{table}
\pagebreak
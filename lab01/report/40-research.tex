\section{Экспериментальная часть}
В данном разделе будут приведены замеры процессорного времени работы функций, а также
проведен сравнительный анализ алгоритмов.
\subsection{Технические характеритики}
Операционная система - \textbf{Ubuntu 22.04 LTS}\cite{ubuntu}.
Процессор - \textbf{AMD Ryzen 5 3500U}.
Оперативная память - 16 Гб.

При тестировании ноутбук был включен в сеть электропитания. Во время тестирования ноутбук был нагружен только встроенными приложениями
окружения, а также системой тестирования.
\subsection{Замеры процессорного времени}
Для измерения процессорного времени используется функция \textbf{clock()} из стандартной библиотеки \textbf{time.h}.
Найти время в секундах позволяет конструкция:
\begin{code}
	\captionsetup{justification=centering}
	\captionof{listing}{Пример использования функции clock()}
	\label{code:5}
	\inputminted
	[
	frame=single,
	framerule=0.5pt,
	framesep=20pt,
	fontsize=\small,
	tabsize=4,
	linenos,
	numbersep=5pt,
	xleftmargin=10pt,
	]
	{text}
	{code/time_example.c}
\end{code}

Замеры проводились 100 раз для всех функций кроме Дамерау-Левенштейна с рекурсией, для него число замеров равно 5, так как данный алгоритм заметно проигрывает в эффективности остальным.
Результаты замеров приведены в таблице 2 (время в секундах).
\pagebreak
\begin{table}[!h]
	\captionsetup{justification=raggedright}
    \caption{Замеры времени}
    \begin{tabular}{|l|l|l|l|l|}
    \hline
        Дл. строки & Л. & Д-Л. & Д-Л.(рек) & Д-Л (рек+кэш) \\ \hline
        0 & 0,000001 & 0,000001 & 0,000001 & 0,000001 \\ \hline
        1 & 0,000001 & 0,000001 & 0,000001 & 0,000001 \\ \hline
        2 & 0,000001 & 0,000001 & 0,000001 & 0,000002 \\ \hline
        3 & 0,000001 & 0,000001 & 0,000002 & 0,000002 \\ \hline
        4 & 0,000002 & 0,000001 & 0,000006 & 0,000003 \\ \hline
        5 & 0,000002 & 0,000002 & 0,000039 & 0,000004 \\ \hline
        6 & 0,000003 & 0,000002 & 0,000201 & 0,000006 \\ \hline
        7 & 0,000003 & 0,000002 & 0,000994 & 0,000007 \\ \hline
        8 & 0,000004 & 0,000004 & 0,006436 & 0,000008 \\ \hline
        9 & 0,000004 & 0,000005 & 0,015010 & 0,000008 \\ \hline
        10 & 0,000004 & 0,000005 & 0,053566 & 0,000007 \\ \hline
        11 & 0,000004 & 0,000005 & 0,291436 & 0,000008 \\ \hline
        12 & 0,000005 & 0,000005 & 1,624380 & 0,000009 \\ \hline
        13 & 0,000005 & 0,000006 & 9,086192 & 0,000009 \\ \hline
    \end{tabular}
\end{table}

Также на рисунках 5-8 приведены графические результаты замеров.

\begin{figure}[!h]
	\centering
	\captionsetup{justification=centering}
    \def\svgscale{0.45}
	\includesvg[]{img/dia1.svg}
	\caption{Сравнение по времени всех алгортимов}
	\label{fig:r1}
\end{figure}

\begin{figure}[!h]
	\centering
	\captionsetup{justification=centering}
    \def\svgscale{0.5}
	\includesvg[]{img/dia2.svg}
	\caption{Сравнение по времени нерекурсивных алгортимов поиска расстояний Левенштейна и Дамерау-Левенштейна}
	\label{fig:r2}
\end{figure}

\begin{figure}[!h]
	\centering
	\captionsetup{justification=centering}
    \def\svgscale{0.5}
	\includesvg[]{img/dia3.svg}
	\caption{Сравнение по времени алгортимов поиска расстояния Дамерау-Левенштейна}
	\label{fig:r3}
\end{figure}

\begin{figure}[!h]
	\centering
	\captionsetup{justification=centering}
    \def\svgscale{0.45}
	\includesvg[]{img/dia4.svg}
	\caption{Сравнение по времени нерекурсивного и рекусивного с кешем алгортимов поиска расстояния Дамерау-Левенштейна}
	\label{fig:r4}
\end{figure}
\pagebreak
\pagebreak


\subsection{Выводы}
Исходя из замеров по памяти, итеративные алгоритмы проигрывают рекурсивным, 
потому что максимальный размер памяти в них растет, как произведение длин строк, а в рекурсивных - как сумма длин строк.

В результате эксперимента было получено, что при длине строк более 10 символов рекусивный алгортим поиска выполнятся намного
дольше других. Его не следует использовать уже при длине строк в 3 символа, так как уже при этой длине он алгоритм медленнее остальных в 3 раза.

Также при проведении эксперимента было выявлено, что нерекурсивные алгортимы поиска расстояний Левенштейна и Дамерау-Левенштейна
имеют примерно одинаковую скорость выполнения, поэтому выбирать между ними нужно исходя из задачи.

Рекурсивный алгортим поиска расстояния Дамерау-Левенштейна с кешем проигрывает нерекурсивному по времени, но выигрывает по памяти.
Осуществлять выбор между этими алгоритмами нужно исходя из того, что важнее - память или время.

\pagebreak
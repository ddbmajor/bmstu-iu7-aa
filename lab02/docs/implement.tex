\chapter{Технологическая часть}

В данном разделе будут рассмотрены средства реализации, а также представлены листинги алгоритмов умножения матриц - стандартный, Винограда и оптимизированный алгоритм Винограда.

\section{Средства реализации}
В данной работе для реализации был выбран язык программирования $Python \cite{python-lang}$. В текущей лабораторной работе требуется замерить процессорное время для выполняемой программы, а также построить графики. Все эти инструменты присутствуют в выбранном языке программирования.

Время работы было замерено с помощью функции \textit{process\_time(...)} из библиотеки $time \cite{python-lang-time}$.


\section{Листинги кода}

В листингах \ref{lst:stand_alg}-\ref{lst:opt_vin_alg} представлены реализации алгоритмов умножения матриц - стандартного, Винограда и оптимизированного алгоритма Винограда.

\begin{center}
    \captionsetup{justification=raggedright,singlelinecheck=off}
    \begin{lstlisting}[label=lst:stand_alg,caption=Стандартный алгоритм умножения матриц]
def standart_alg(mat_a, mat_b):
	n = len(mat_a)
	m = len(mat_a[0])
	t = len(mat_b[0])
	res_matrix = [[0] * t for _ in range(n)]

	for i in range(n):
		for j in range(m):
			for k in range(t):
				res_matrix[i][j] += mat_a[i][k] * mat_b[k][j]

	return res_matrix
\end{lstlisting}
\end{center}


\begin{center}
    \captionsetup{justification=raggedright,singlelinecheck=off}
    \begin{lstlisting}[label=lst:vin_alg,caption=Алгоритм Винограда умножения матриц]
def vinograd_alg(mat_a, mat_b):

	n = len(mat_a)
	m = len(mat_a[0])
	t = len(mat_b[0])

	res_matrix = [[0] * t for _ in range(n)]

	tmp_row = [0] * n
	tmp_col = [0] * t

	for i in range(n):
		for j in range(0, m // 2):
			tmp_row[i] = tmp_row[i] + mat_a[i][2 * j] * mat_a[i][2 * j + 1]

	for i in range(t):
		for j in range(0, m // 2):
			tmp_col[i] = tmp_col[i] + mat_b[2 * j][i] * mat_b[2 * j + 1][i] 


	for i in range(n):
		for j in range(t):

			res_matrix[i][j] = -tmp_row[i] - tmp_col[j]   

			for k in range(0, m // 2):

				res_matrix[i][j] = res_matrix[i][j] + (mat_a[i][2 * k + 1] + mat_b[2 * k][j]) * (mat_a[i][2 * k] + mat_b[2 * k + 1][j])

	if (m % 2 == 1):
		for i in range(n):
			for j in range(t):
				res_matrix[i][j] = res_matrix[i][j] + mat_a[i][m - 1] * mat_b[m - 1][j]

	return res_matrix
\end{lstlisting}
\end{center}


\begin{center}
    \captionsetup{justification=raggedright,singlelinecheck=off}
    \begin{lstlisting}[label=lst:opt_vin_alg,caption=Оптимизированный алгоритм Винограда умножения матриц]
def optimized_vinograd_alg(mat_a, mat_b):

	n = len(mat_a)
	m = len(mat_a[0])
	t = len(mat_b[0])

	res_matrix = [[0] * t for _ in range(n)]

	tmp_row = [0] * n
	tmp_col = [0] * t

	for i in range(n):
		for j in range(0, m // 2):
			tmp_row[i] += mat_a[i][j << 1] * mat_a[i][(j << 1) + 1]

	for i in range(t):
		for j in range(0, m // 2):
			tmp_col[i] += mat_b[j << 1][i] * mat_b[(j << 1) + 1][i]

	for i in range(n):
		tmp = -tmp_row[i]

		for j in range(t):

			res_matrix[i][j] = tmp - tmp_col[j]

			for k in range(0, m // 2):
				res_matrix[i][j] += (mat_a[i][(k << 1) + 1] + mat_b[k << 1][j]) * (
							mat_a[i][k << 1] + mat_b[(k << 1) + 1][j])

	if m % 2 == 1:
		m_min_one = m - 1
		for i in range(n):
			tmp = mat_a[i][m_min_one]
			for j in range(t):
				res_matrix[i][j] += tmp * mat_b[m_min_one][j]

	return res_matrix
\end{lstlisting}
\end{center}


\section{Функциональные тесты}

В таблице \ref{tbl:functional_test} приведены тесты для функций, реализующих алгоритмы умножения матриц, рассматриваемых в данной лабораторной работе. Тесты \textit{для всех алгоритмов} пройдены успешно.


\begin{table}[h]
	\begin{center}
		\begin{threeparttable}
		\captionsetup{justification=raggedright,singlelinecheck=off}
		\caption{\label{tbl:functional_test} Функциональные тесты}
		\begin{tabular}{|c@{\hspace{7mm}}|c@{\hspace{7mm}}|c@{\hspace{7mm}}|c@{\hspace{7mm}}|c@{\hspace{7mm}}|c@{\hspace{7mm}}|}
			\hline
			Матрица 1 & Матрица 2 & Ожидаемый результат \\ 
			\hline

			$\begin{pmatrix}
				1 & 5 & 7\\
				2 & 6 & 8\\
				3 & 7 & 9
			\end{pmatrix}$ &
			$\begin{pmatrix}
				&
			\end{pmatrix}$ &
			Сообщение об ошибке \\ \hline

			$\begin{pmatrix}
				1 & 2 & 3
			\end{pmatrix}$ &
			$\begin{pmatrix}
				1 & 2 & 3
			\end{pmatrix}$ &
			Сообщение об ошибке \\ \hline

			$\begin{pmatrix}
				1 & 1 & 1
			\end{pmatrix}$ &
			$\begin{pmatrix}
				1 \\
				1 \\
				1
			\end{pmatrix}$ &
			$\begin{pmatrix}
				1 & 1 & 1\\
				1 & 1 & 1 \\
				1 & 1 & 1
			\end{pmatrix}$ \\ \hline

			$\begin{pmatrix}
				1 & 2 & 3 \\
				4 & 5 & 6 \\
				7 & 8 & 9
			\end{pmatrix}$ &
			$\begin{pmatrix}
				1 & 0 & 0 \\
				0 & 1 & 0 \\
				0 & 0 & 1
			\end{pmatrix}$ &
			$\begin{pmatrix}
				1 & 2 & 3 \\
				4 & 5 & 6 \\
				7 & 8 & 9
			\end{pmatrix}$ \\ \hline

			$\begin{pmatrix}
				2
			\end{pmatrix}$ &
			$\begin{pmatrix}
				2
			\end{pmatrix}$ &
			$\begin{pmatrix}
				4
			\end{pmatrix}$ \\ \hline

		\end{tabular}
		\end{threeparttable}
	\end{center}
	
\end{table}

\section{Вывод}

Были представлены листинги всех алгоритмов умножения матриц - стандартного, Винограда и оптимизированного алгоритма Винограда. Также в данном разделе была приведена информации о выбранных средствах для разработки алгоритмов.

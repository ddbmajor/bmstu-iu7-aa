\chapter{Аналитическая часть}
В этом разделе будут представлены классический алгоритм умножения матриц и алгоритм Винограда.

\section{Стандартный алгоритм}

Пусть даны две матрицы

\begin{equation}
	A_{lm} = \begin{pmatrix}
		a_{11} & a_{12} & \ldots & a_{1m}\\
		a_{21} & a_{22} & \ldots & a_{2m}\\
		\vdots & \vdots & \ddots & \vdots\\
		a_{l1} & a_{l2} & \ldots & a_{lm}
	\end{pmatrix},
	\quad
	B_{mn} = \begin{pmatrix}
		b_{11} & b_{12} & \ldots & b_{1n}\\
		b_{21} & b_{22} & \ldots & b_{2n}\\
		\vdots & \vdots & \ddots & \vdots\\
		b_{m1} & b_{m2} & \ldots & b_{mn}
	\end{pmatrix},
\end{equation}

тогда матрица $C$
\begin{equation}
	C_{ln} = \begin{pmatrix}
		c_{11} & c_{12} & \ldots & c_{1n}\\
		c_{21} & c_{22} & \ldots & c_{2n}\\
		\vdots & \vdots & \ddots & \vdots\\
		c_{l1} & c_{l2} & \ldots & c_{ln}
	\end{pmatrix},
\end{equation}

где
\begin{equation}
	\label{eq:M}
	c_{ij} =
	\sum_{r=1}^{m} a_{ir}b_{rj} \quad (i=\overline{1,l}; j=\overline{1,n})
\end{equation}

будет называться произведением матриц $A$ и $B$.

Стандартный алгоритм реализует данную формулу.


\section{Алгоритм Винограда}

\textbf{Алгоритм Винограда} \cite{vinograd-matrix} — алгоритм умножения квадратных матриц. Начальная версия имела асимптотическую сложность алгоритма примерно $O(n^{2,3755})$, где $n$ - размер стороны матрицы, но после доработки он стал обладать лучшей асимптотикой среди всех алгоритмов умножения матриц.

Рассмотрим два вектора $V = (v_1, v_2, v_3, v_4)$ и $W = (w_1, w_2, w_3, w_4)$.
Их скалярное произведение равно: $V \cdot W = v_1w_1 + v_2w_2 + v_3w_3 + v_4w_4$, что эквивалентно:
\begin{equation}
	\label{for:new}
	V \cdot W = (v_1 + w_2)(v_2 + w_1) + (v_3 + w_4)(v_4 + w_3) - v_1v_2 - v_3v_4 - w_1w_2 - w_3w_4.
\end{equation}

Прирост производительности заключается в идее предварительной обработки - полученное выражение требует большего количества операций, чем стандартное умножение матриц, но выражение в правой части крайнего равенства можно вычислить заранее и запомнить для каждой строки первой матрицы и каждого столбца второй матрицы. 
Это позволит выполнить лишь два умножения и девять сложений, при учете, что потом будет сложено только с двумя предварительно посчитанными суммами соседних элементов текущих строк и столбцов. Операция сложения выполняется быстрее, поэтому на практике алгоритм должен работать быстрее обычного алгоритма перемножения матриц.

Но стоит упомянуть, что при нечетном значении размера матрицы нужно дополнительно добавить произведения крайних элементов соответствующих строк и столбцов.

\section[Оптимицированный алгоритм Винограда]{Оптимицированный алгоритм\\Винограда}
Оптимизация заключается в:
\begin{itemize}
    \item использовании побитового сдвига вместо умножения на 2;
    \item предвычисление некоторых слагаемых;
    \item операция сложения заменена на операцию $+=$. \newline
\end{itemize}

\section{Вывод}

В данном разделе были рассмотрены алгоритмы умножения матриц - стандартного и Винограда, который имеет большую эффективность за счет предварительных вычислений.

Алгоритмы на вход будут получать две матрицы, при чем количество столбцов одной матрицы должно совпадать с количеством строк второй матрицы. При вводе пустой матрицы будет выведено сообщение об ошибке. Реализуемое ПО дает возможность выбрать один из алгоритмов или все сразу, ввести две матрицы и вывести результат перемножить, а также можно провести тестирование по времени для четных и нечетных размеров матриц и сравнить результаты, используя их графическое представление.
